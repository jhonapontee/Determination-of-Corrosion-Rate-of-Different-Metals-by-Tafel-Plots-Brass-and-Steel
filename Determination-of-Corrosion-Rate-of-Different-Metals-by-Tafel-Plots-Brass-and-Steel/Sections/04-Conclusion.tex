% Conclusion
\section{Conclusion}

Electrochemical analysis using Tafel curves in a 0.5 M HCl medium demonstrated that brass possesses significantly higher corrosion resistance than steel. This is evidenced by its significantly lower corrosion current density $(5.40 * 10^{-6} A/cm^2)$ compared to that of steel $(8.87 * 10^{-5} A/cm^2)$. The superiority of brass is confirmed by calculating the Corrosion Penetration Rate (CPR), which was $4.17 * 10^{-5} mm/year$ for brass, a much lower value than the CPR of steel $(8.23 * 10^{-4} mm/year)$. This practical metric indicates a much slower annual thickness loss. Additionally, the corrosion potential $(E_{corr})$ of brass (-0.258 V) was less negative than that of steel (-0.388 V), reflecting a lower thermodynamic tendency to undergo oxidation. Therefore, if a material needs to be exposed to an acidic environment, brass is much more efficient than steel due to its corrosion resistance.

